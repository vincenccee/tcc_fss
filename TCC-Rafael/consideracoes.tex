\chapter{Considerações}
\label{ch:consideracoes}

De um modo geral, a maioria das pesquisas em algoritmos evolutivos concentra-se em problemas de otimização estáticos. No entanto, muitos problemas do mundo real são problemas de optimização dinâmica, em que as mudanças ocorrem ao longo do tempo. Isso requer algoritmos de otimização, não só para encontrar a solução ideal global sob um ambiente específico, mas também para monitorar continuamente a mudança em diferentes ambientes dinâmicos. Assim, são necessários métodos de optimização que são capazes de se adaptar continuamente para um ambiente em mudança.

Os algoritmos de inspiração biológica, em principal, os algoritmos de inteligência de enxame são amplamente estudados para melhorar sua performance em ambientes dinâmicos e com isso vários operadores evolutivos são gerados e aplicados em diferentes problemas dinâmicos com diferentes propriedades. Cada um desses operadores possuem pontos fortes e fracos nas suas aplicações, então o estudo comparativo para identificar esse pontos se faz necessário, de modo que a aplicação em conjunto desses operadores pode se benéfico para ambos, neutralizando esses pontos negativos.


