\chapter{Introdução}
\label{ch:intro}

Na sociedade existem vários tipos de problemas que possuem uma grande dificuldade de obterem resultados satisfatórias em um tempo factível. Esses problemas tendem a ter uma grande dimensionalidade e estarem sempre se alterando, o que os tornas problemas complexos de serem otimizados \cite{de2004otimizaccao}. Na Natureza pode-se observar também uma constante mudança no ambiente, de forma que seus integrantes tenham que se adaptar a essas mudanças para poderem sobreviver. Sendo assim, na Natureza existe uma grande fonte de inspiração para o desenvolvimento de novas tecnologias para solucionar essa classe de problemas, particularmente na Ciência da Computação, dentro da área de Computação Natural, que é responsável pela criação de diversos algoritmos \cite{de2007fundamentals}. A partir da observação e compreensão do comportamento de animais ou colônia de animais (fenômenos naturais) foram desenvolvidas várias tecnologias em diferentes aplicações \cite{rozenberg2011handbook}, pois a natureza é capaz de trabalhar com problemas de alta complexidade e grande dimensionalidade, motivando o desenvolvimento de algoritmos bioinspirados na área de otimização \cite{andre2015multiple}.

A característica que vem chamando atenção para a área de otimização na computação é justamente o fato dos problemas estarem em contante alteração e uma solução ótima, que foi encontrada em um instante de tempo anterior, pode não ser mais suficientemente boa para o mesmo problema no futuro. Isso é uma questão muito importante, pois a maioria dos algoritmos usados na otimização de problemas leva muito tempo para encontrar uma solução factível, por uma busca exaustiva e, se em pouco tempo ela perder a validade, o sistema de otimização não pode ser aplicado em ambientes dinâmicos \cite{morrison2003performance}. Por esse motivo, para otimizar esses problemas são utilizados os algoritmos bioinspirados, que por sua vez possuem vários operadores e características relevantes que podem ser aplicados e estudados em separado e assim estipular a relevância de cada um.

Na literatura são encontrados diversos exemplos de algoritmos com inspiração natural, que podem se basear no comportamento de uma colônia de animais por busca de alimento, na interação que eles podem realizar entre si em sua colônia, ou até na própria evolução das espécies. Uma das primeiras meta-heurísticas inspiradas na natureza, em especial na Biologia, são os Algoritmos Genéticos (\textit{Genetic Algorithms}) \cite{holland1975adaptation}. Esta abordagem se baseia na teoria da evolução proposta por Darwin em que os indivíduos mais bem adaptados tem maiores chances de sobreviver e passar seu material genético adiante, essa classe de algoritmos é chamada de computação evolucionária (\textit{Evolutionary Algorithms} - EA) que possuem rotinas específicas. Outro algoritmo que entra nessa classe é o de Evolução Diferencial (\textit{Diferential Evolution} - DE), que também possui uma rotina de seleção, cruzamento e mutação. 

Dentre outros algoritmos bioinspirados existe uma classe que se baseia no comportamento simples de cada indivíduo separadamente e que em conjunto pode-se gerar um comportamento mais complexo, ou seja, um comportamento emergente. E essa classe é chamada de algoritmos de Inteligência de Enxame (\textit{Inteligence Swarm} - IS) \cite{parpinelli2011new}. Entre eles pode-se citar algoritmos que tem sua aplicação a problemas contínuos, como o algoritmo baseado no comportamento de uma colônia de bactérias na busca por alimentos, o \textit{Biomimicry of Bacterial Foraging Algorithm} (BFA) \cite{passino2002biomimicry}, o algoritmo baseado no comportamento das colônias de vaga-lumes e sua bioluminecência que guia os outros vaga-lumes no espaço de busca, o \textit{Firefly Algorthm} \cite{firefly}. O algoritmo inspirado no comportamento coordenado do movimento de cardumes de peixes e revoada de pássaros, o Algoritmo de Otimização por Enxame de Partículas (\textit{Partical Swarm Optimization}, PSO) \cite{pso}, dentre outros.

Um algoritmo que será o foco do trabalho, é o algoritmo baseado no comportamento de um cardume de peixes, o \textit{Fish School Search Optimization} (FSS) \cite{carmelo2008novel}, pois possuí operadores evolutivos eficientes na manutenção da diversidade quando é percebido uma piora na evolução e, na intensificação caso contrário. Suas principais características estão nos seus operadores evolutivos e na influência que cada um deles tem no processo de otimização \cite{c2009influence}, sendo eles: Operador de movimento individual; Operador de alimentação; Operador de movimento coletivo instintivo; e o operador de movimento volátil coletivo. O operador de movimento volátil coletivo tem uma influência maior na resolução de problemas dinâmicos e contínuos, pelo fato de expandir e contrair a busca do cardume de peixes, dependendo do nível de melhoramento das soluções em relação a experiências recentes, o que ajuda na manutenção da resiliência da busca. Outro modelo desse algoritmo possui outros operadores evolutivos \cite{madeiro2011density}, que são: Operador de memória; e o Operador de divisão da escola de peixes. O operador de movimentação volátil foi aplicado no PSO para melhorar o desempenho em problemas dinâmicos, sendo criado uma nova versão, o \textit{Volitive} PSO \cite{cavalcanti2011hybrid}, que obteve bons resultados aplicado a essa classe de problemas.

A literatura apresenta uma vasta quantidade de trabalhos que estudam a aplicação desses algoritmos bioinspirados em problemas dinâmicos com domínio contínuo. Entre eles pode-se citar o AG, que no trabalho \cite{rand2005measurements} foi analisado seus componentes em separado, não somente na sua performance. A aplicação do Algoritmo de Evolução diferencial com as versões baseadas em aglomeração (\textit{Crowding-based DE} CDE), baseado em compartilhamento (\textit{Sharing-based DE} ShDE) \cite{thomsen2004multimodal}, e a versão baseada em espécies (\textit{Species-based DE} SDE) \cite{li2005efficient}. Essas versões do DE foram estudadas e o \textit{Crowding-based local Differential Evolution with Speciation-based Memory} (ClDES) foi desenvolvido utilizando seus postos positivos. Na área de Inteligência de enxames é encontrado várias versões do PSO, como por exemplo: \textit{Dynamic Species-Based Particle Swarm Optimizer} DSPSO \cite{parrott2006locating} e o \textit{Clustering Particle Swarm Optimizer} CPSO \cite{yang2010clustering}. Existe também os algortimos de (\textit{Dynamic Bacterial Foraging Algorithm} - DBFA) proposto por \cite{passino2002biomimicry} e o \textit{Multiswarm Based Firefly Algorithm} proposto por \cite{farahani2011multiswarm}.

Como não se pode ter uma solução ótima a todo momento em problemas dinâmicos, o tempo de execução é considerado como uma unidade discreta. Geralmente um dos fatores limitantes em uma aplicação acaba sendo o tempo de execução sendo necessário manter um equilíbrio entre a qualidade da solução e o tempo de execução do algoritmo \cite{li2006new}. Apesar dos algoritmos bioinspirados serem capazes de encontrar boas soluções para problemas reais, eles tendem a perder a eficiência quando aplicados a problemas de larga escala. Esta característica indesejável é conhecida por “maldição da dimensionalidade” (\textit{curse of dimensionality}) \cite{bellman2015applied}. Isso ocorre devido ao crescimento exponencial do espaço de busca de acordo com as dimensões do problema.

O grande número de dimensões aumenta a dificuldade dos algoritmos em manter soluções aceitáveis. A qualidade da otimização de um algoritmo depende do equilíbrio dos componentes de diversificação e intensificação \cite{boussaid2013survey}, pois a diversificação é responsável pela exploração do espaço de busca como um todo e a intensificação é responsável pela acurácia da resposta. A natureza evoluiu para manter o equilíbrio de diversidade e na otimização é possível usar alguns componentes de controle para sua preservação. A literatura aponta diversas estratégias para este controle de diversificação, sendo algumas: \textit{fitness sharing, clearing, crowding, deterministic crowding, probabilistic crowding} e \textit{restricted} \cite{andre2015multiple}.

Na literatura, os algoritmos bioinspirados são aplicados a diversos tipos de problemas dinâmicos, sendo eles contínuos ou discretos, então para poder avaliar os algoritmos, utiliza-sa diversas funções \textit{benchmarks} \cite{moser2007review}, como por exemplo a função de Movimentação de Picos (\textit{Moving Peaks} - MP). A qualidade da solução sendo otimizada pode ser mensurada pela sua aptidão, ou seja, quão interessante é o valor encontrado e quão bem o algoritmo se adapta a uma mudança do ambiente. Assim, uma solução com uma boa aptidão (\textit{fitness}) durante o processo de evolução é considerada válida e é chamada de solução factível. Durante a otimização dos \textit{benchmarks} podem aparecer diferentes tipos de dinamismo, como por exemplo: na função objetivo; no domínio das variáveis; no número de variáveis; nas restrições; ou outros parâmetros.

Na realização desse trabalho foram estipulados objetivos para determinar os passos a serem seguidos, na próxima seção serão apresentados esses objetivos para estruturar os capítulos do trabalho.	

\section{Objetivo}
\label{sec:objetivo}
%contribuição
Este trabalho tem como objetivo analisar da eficiência do FSS em problemas dinâmicos com domínio contínuos e com alta dimensionalidade. A proposta central é analisar cada um dos operados evolutivos dos algoritmos e determinar a relevância de cada um para a otimização dessa classe de problemas. Para isso será feito uma análise comparativa dos algoritmos que tem relevância na otimização desses problemas, indicando os pontos positivos e negativos de cada um. Para atingir o objetivo principal alguns objetivos específicos foram traçados:

\begin{itemize}
\item Revisão bibliográfica dos conceitos da computação natural e meta-heurísticas bioinspiradas;
\item Análise aprofundada do FSS e seus operadores;
\item Revisão bibliográfica dos operadores evolutivos existentes;
\item Estudo dos algoritmos evolutivos de inteligência de enxame para verificar a relevância dos operadores existentes e realizar uma análise comparativa
\item Levantamento e descrição de funções dinâmicos para serem aplicados nos experimentos;
\item Desenvolver um algoritmo utilizando os operadores evolutivos alternadamente;
\item Realizar experimentos computacionais com o algoritmo e os problemas selecionados;
\item Coleta e análise dos resultados dos experimentos.
\end{itemize} 

\section{Estrutura do Trabalho}
\label{sec:escopo}

O trabalho está organizado em 6 Capítulos, incluindo a introdução

No segundo capítulo é feita uma revisão da definição de intensificação e diversificação, em seguida é feito uma revisão do estado da arte dos algoritmos bioinspirados e as suas características principais, sendo separados em dois grupos: os algoritmos evolutivos e os de inteligência de enxames. Por fim, é apresentado uma revisão da divesidade populacional e Métricas encontradas na literatura.

No terceiro capítulo é apresentada uma definição dos problemas que serão estudados neste trabalho, esquematizando suas características principais, suas dificuldades e métodos de avaliação para o algoritmo que serà utilizado para realizar a otimização.

No quarto capítulo é apresentada uma revisão dos trabalhos encontrados nessa área de pesquisa, onde pode ser analisado outras aplicações em problemas dinâmicos e com isso definir quais os pontos mais relevantes.

No quinto capítulo é descrita a dinâmica da proposta e as implementações necessárias para concretiza-la.

Por fim, no sexto capítulo encerra-se o trabalho com uma breve revisão das principais considerações, apresenta uma discussão dos resultados obtidos com a pesquisa e aponta trabalhos futuros.