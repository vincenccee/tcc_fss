\chapter{Introdução}
\label{ch:intro}

Na sociedade existem vários tipos de problemas que possuem uma grande dificuldade de obterem resultados satisfatórias em um tempo factível, esses problemas tendem a ter uma grande dimensionalidade, serem complexos e estarem sempre se alterando, se adaptando ao meio \cite{de2004otimizaccao}. Na natureza pode-se observar também, uma constante mudança no ambiente, de forma que seus integrantes tenham que se adaptar, a essas mudanças, para poderem sobreviver, sendo assim, nela há uma grande fonte de inspiração no desenvolvimento de novas tecnologias para solucionar essa classe de problemas, particularmente na Ciência da Computação, dentro da área de Computação Natural, que é responsável pela criação de diversos algoritmos \cite{de2007fundamentals}. A partir da observação e compreensão do comportamento de animais ou colônia de animais (fenômenos naturais) foram desenvolvidas várias tecnologias em diferentes aplicações \cite{rozenberg2011handbook}, pois a natureza é capaz de trabalhar com problemas de alta complexidade e grande dimensionalidade, motivando o desenvolvimento de algoritmos bioinspirados na área de otimização \cite{andre2015multiple}.

A característica que vem chamando atenção para a área de otimização na computação, é justamente o fato dos problemas estarem em contante alteração e uma solução ótima, que foi encontrada no passado, pode não ser mais suficiente boa para o mesmo problema no futuro. Isso é uma questão muito importante, pois a maioria dos algoritmos usados na otimização de problemas leva muito tempo para encontrar uma solução factível, por uma busca exaustiva, e se em pouco tempo ela perder a validade, o sistema de otimização não pode ser aplicado em ambientes dinâmicos \cite{morrison2003performance}. Por esse motivo, para otimizar esses problemas, com domínio contínuo, são utilizados os algoritmos bioinspirados, que por sua vez, possuem vários operadores e características relevantes, que podem ser aplicados e estudas em separado e assim estipular a relevância de cada um para cara tipo de problema.

Na literatura são encontrados diversos exemplos de algoritmos com inspiração natural, que podem se basear no comportamento de uma colônia de animais por busca de alimento, na interação que eles podem realizar entre si em sua colônia, ou até na própria evolução das espécies. Uma das primeiras meta-heurísticas inspiradas na natureza, em especial na Biologia, é o algoritmo genético (\textit{Genetic Algorithm}) \cite{holland1975adaptation}, esta abordagem se baseia na teoria da evolução proposta por Darwin, em que os indivíduos mais adaptados tem maiores chances de sobreviver e passar seu material genético adiante. Dentre outros algoritmos bioinspirados existe uma classe de algoritmos que se baseia na interação de indivíduos de mesma espécie em uma colônia, essa classe é chamada de algoritmos de Inteligência de Enxame \cite{parpinelli2011new}. Entre eles pode-se citar algoritmos que tem sua aplicação a problemas contínuos, como o algoritmo baseado no comportamento de uma colônia de bactérias na busca por alimentos, o \textit{Biomimicry of bacterial foraging} (BFO) \cite{passino2002biomimicry}, o algoritmo baseado no comportamento das abelhas na busca de alimentos no Algoritmo de Colônia Artificial de Abelhas (\textit{Artificial Bee Colony}, ABC) \cite{abc}. o algoritmo inspirado no comportamento coordenado do movimento de cardumes de peixes e revoada de pássaros, o algoritmo de otimização por enxame de partículas (\textit{Partical Swarm Optimization}, PSO) \cite{pso}, possuindo também uma outra versão, em que foi incluído um novo operador, tendo a possibilidade de ter partículas carregadas para ajudar na manutenção da diversidade genética \cite{modified_pso}. O PSO possuí estudos realizados, utilizando suas várias versões, com vários operadores evolutivos diferentes, aplicados em ambientes dinâmicos contínuos \cite{carlisle2002applying}.

Um algoritmo, com características relevantes para ser aplicado na solução de problemas dinâmicos com domínio contínuo, que será o foco do trabalho, é o algoritmo baseado no comportamento de um cardume de peixes, o \textit{Fish School Search Optimization} (FSS) \cite{carmelo2008novel}. Sua principal característica são seus operadores evolutivos e a influência que cada um deles tem no processo de otimização \cite{c2009influence}, sendo eles: Operador de movimento individual; Operador de alimentação; Operador de movimento instintivo coletivo; E o operador de movimento volátil coletivo. O operador de movimento volátil coletivo tem uma influência maior na resolução de problemas dinâmicos e contínuos, pelo fato de expandir e contrair a busca do cardume de peixes, dependendo do nível de melhoramento das soluções em relação a experiências recentes, o que ajuda na manutenção da resiliência da busca. Outro modelo desse algoritmo possui outros operadores evolutivos \cite{madeiro2011density}, que são: Operador de memória; e o Operador de divisão da escola de peixes. O operador de movimentação volátil foi aplicado no PSO para melhorar o desempenho em problemas dinâmicos, sendo criado uma nova versão, o \textit{Volitive} PSO \cite{cavalcanti2011hybrid}, que obteve bons resultados aplicado a essa classe de problemas.

A literatura apresenta uma vasta quantidade de trabalhos que estudam a aplicação desses algoritmos bioinspirados em problemas dinâmicos com domínio contínuo. Entre eles pode-se citar o ****

Como não se pode ter uma solução ótima a todo momento, o tempo de execução é considerado como uma unidade discreta. Geralmente um dos fatores limitantes em uma aplicação acaba sendo o tempo de execução, então é necessário manter um equilíbrio entre o qualidade da solução e o tempo de execução do algoritmo \cite{li2006new}. Apesar dos algoritmos bioinspirados serem capazes de encontrar boas soluções para problemas reais, eles tendem a perder à eficiência quando aplicados a problemas de larga escala. Esta característica indesejável é conhecida por “maldição da dimensionalidade” (\textit{curse of dimensionality}) \cite{bellman2015applied}. Isso ocorre devido ao crescimento exponencial do espaço de busca de acordo com as dimensões do problema.

O grande número de dimensões aumenta a dificuldade dos algoritmos em manter soluções aceitáveis. A qualidade da otimização de um algoritmo depende do equilíbrio dos componentes de diversificação e intensificação \cite{boussaid2013survey}, pois a diversificação é responsável pela exploração do espaço de busca como um todo, e a intensificação é responsável pela acurácia da resposta. A natureza evoluiu para manter o equilíbrio de diversidade e na otimização é possível usar alguns componentes de controle para sua preservação. A literatura aponta diversas estratégias para este controle de diversificação, sendo algumas: \textit{fitness sharing, clearing, crowding, deterministic crowding, probabilistic crowding} e \textit{restricted} \cite{andre2015multiple}.

Na literatura, os algoritmos bioinspirados são aplicados diversos tipos de problemas dinâmicos, sendo eles contínuos ou discretos, então para poder avaliar os algoritmos, a literatura utiliza diversas funções \textit{benchmarks} \cite{moser2007review}. A qualidade da solução sendo otimizada pode ser mensurada pela sua aptidão, ou seja, quão interessante é o valor encontrado, e quão bem o algoritmo se adapta a uma mudança do ambiente. Assim, uma solução com uma boa aptidão (\textit{fitness}) durante o processo de evolução, é considerada valida e é chamada de solução factível. Durante a otimização dos \textit{benchmarks} podem aparecer diferentes tipos de dinamismo, como por exemplo: na função objetivo; no domínio das variáveis; no número de variáveis; nas restrições; ou outros parâmetros.


\section{Objetivo}
\label{sec:objetivo}
%contribuição
Este trabalho tem como contribuição analisar da eficiência do FSS em problemas reais, dinâmicos com domínio contínuos e com dimensionalidade alta. A proposta central é analisar cada um dos operados evolutivos, encontrados na literatura, e determinar a relevância de cada um para a otimização dessa classe de problemas. Para isso será feito uma análise comparativa dos algoritmos que tem relevância na otimização desses problemas, indicando os pontos positivos de cada um aplicando no FSS e com isso foram estipulados alguns objetivos específicos:

\begin{itemize}
\item Revisão bibliográfica dos conceitos da computação natural e meta-heurísticas bioinspiradas;
\item Análise aprofundada do FSS e seus operadores.
\item Revisão bibliográfica dos operadores evolutivos existentes;
\item Estudo dos algoritmos evolutivos de inteligência de enxame para verificar a relevância dos operadores existentes e realizar uma análise comparativa;
\item Estudo de técnicas de intensificação e diversificação para testar a influência na qualidade da busca em problemas dinâmicos;
\item Levantamento e descrição de problemas dinâmicos de larga escala que possuam uma grande dimensionalidade para serem aplicados nos experimentos;
\item Desenvolver um algoritmo utilizando os operadores evolutivos alternadamente;
\item Realizar experimentos computacionais com o algoritmo e os problemas selecionados;
\item Coleta e análise dos resultados dos experimentos
\end{itemize} 

\section{Estrutura do Trabalho}
\label{sec:escopo}

O trabalho está organizado em 6 capítulos, incluindo a introdução

No segundo capítulo é apresentado uma definição dos problemas que serão estudados neste trabalho, esquematizando suas características principais, suas dificuldades e um método de avaliação dos algoritmos que será usados para realizar a otimização.

No terceiro capítulo é feito uma revisão do estado da arte dos algoritmos bioinspirados e as suas características principais, sendo separados em dois grupos, os algoritmos evolutivos e os de inteligência de enxames. Em seguida é apresentado uma revisão das técnicas de intensificação e diversificação encontradas na literatura. Por fim uma análise aprofundada do FSS que será o foco deste trabalho.

No Quarto capítulo é apresentado uma revisão dos trabalhos encontrados nessa área de pesquisa, onde pode ser analisado outras aplicações em problemas dinâmicos e com isso definir quais os pontos mais relevantes.

No quinto capítulo é descrito a dinâmica da proposta e as implementações necessárias para concretiza-la.

Por fim, no sexto capítulo encerra-se o trabalho com uma breve revisão das principais considerações, apresenta uma discussão dos resultados obtidos com a pesquisa e aponta trabalhos futuros.