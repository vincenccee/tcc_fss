\usepackage[brazil,brazilian,portuges]{babel}
%\usepackage[brazilian]{babel}
\usepackage{times}
\usepackage{amsmath,listings,indentfirst,url,hyperref}
%\usepackage{amsmath,listings,indentfirst}
\usepackage{paralist}
\usepackage[all]{hypcap}
\usepackage{paralist}
\usepackage{subfigure,comment}
\usepackage[alf]{abntcite}
\usepackage[portuguese]{nomencl}
\usepackage{longtable}
\usepackage{fancyvrb}
\usepackage{color}
\usepackage{rotating}
\usepackage{multirow}
\usepackage{latexsym}
\usepackage [ table ]{xcolor}
\usepackage[utf8]{inputenc}
\usepackage[section]{placeins}
\usepackage{graphicx}


% pacotes de fontes! doi ter que usar times para evitar problemas :(
%\usepackage{fontspec}
%\usepackage{xunicode} 
%\defaultfontfeatures{Mapping=tex-text} 
%\setromanfont{Garamond}
%\setsansfont{Gill Sans}
%\setmonofont{Courier New}

% Listings
\lstset{
    inputencoding=utf8,
    basicstyle=\scriptsize,
    frame=single,
	tabsize=4,
	captionpos=b,
	breaklines=true,
	numbers=left,
}

% Não sei porque mas o LaTeX insiste em avançar a margem nas citações.
% Felizmente há o comando \sloppy, que diz para aumentar o espaçamento
% entre as palavras, visando /sempre/ respeitar as margens. Fica feio
% em alguns parágrafos, mas é a vida...
\sloppy
\begin{comment}
\tolerance 1414
\hbadness 1414
\emergencystretch 1.5em
\hfuzz 0.3pt
\widowpenalty=10000
\vfuzz \hfuzz
\raggedbottom
\end{comment}

% da um trato nos floats
\renewcommand{\topfraction}{.85}
\renewcommand{\bottomfraction}{.7}
\renewcommand{\textfraction}{.15}
\renewcommand{\floatpagefraction}{.66}
\renewcommand{\dbltopfraction}{.66}
\renewcommand{\dblfloatpagefraction}{.66}
\setcounter{topnumber}{9}
\setcounter{bottomnumber}{9}
\setcounter{totalnumber}{20}
\setcounter{dbltopnumber}{9}

\addto\captionsportuges{
  \renewcommand{\tablename}
    {Quadro}}

\addto\captionsportuges{
  \renewcommand{\listtablename}
    {Lista de Quadros}}

\renewcommand{\lstlistingname}{C\'odigo}
\def\listofsymbols{\def\addsymbol #1 #2{#1  & \hspace{0.5in} #2 \\ } 

\begin{tabular}{l l}
	\addsymbol FSS {\textit{Fishing School Shearch}}
	\addsymbol PSO {\textit{Parocle Swarm Optimization}}
	\addsymbol ABC {\textit{Artificial bee Colony}}
	\addsymbol CN {\textit{Computação Natural}}
	\addsymbol IS {\textit{Inteligence Swarm}}
	\addsymbol MP {\textit{Moving Peaks}}
	\addsymbol OP {\textit{Ocillating Peaks}}
	\addsymbol EA {\textit{Evolutionary Algorithms}}
	\addsymbol CN {\textit{Computação Natural}}
	\addsymbol CN {\textit{Computação Natural}}
\end{tabular}

 \clearpage}

% Verbatim não precisa de espaçamento duplo e nem de fontes tão grandes
\RecustomVerbatimEnvironment
	{Verbatim}%
	{Verbatim}%
	{baselinestretch=1,	fontsize=\relsize{-1}}
