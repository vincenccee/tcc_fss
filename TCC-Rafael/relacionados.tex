\chapter{Trabalhos Relacionados}
\label{ch:relacionados}
A Computação Natural possuí vários operadores diferentes em algoritmos diferentes para serem aplicados em diversos tipos de problemas. Nesta Seção são apresentados os trabalhos do estado da arte encontrados na literatura que utilizam esses algoritmos bioinspirados em problemas dinâmicos de domínio contínuo.

\section{Evolução Diferencial Local a Base de Aglomeração e com Memória Baseada em Espécies}
\label{sec:crowding_base_de}

No trabalho de \cite{kundu2013crowding}, é feito uma análise comparativa das versões do DE existentes, especificando suas necessidades ao serem aplicados em ambiente dinâmicos e ao final é apresentado a proposta de aplicação. As versões que são analisadas são as seguintes:

\begin{itemize}
\item Evolução Diferencial Baseada em Aglomeração (\textit{Crowding-based DE} - CDE) \cite{thomsen2004multimodal}: Basicamente, o CDE estende DE com o esquema \textit{Crowding}. Assim, a única modificação ao DE convencional é sobre o indivíduo (pai) ser substituído. Normalmente o pai produzir a descendência é substituído, ao passo que no CDE a descendência substitui o indivíduo mais similar entre um subconjunto da população.

\item Evolução Diferencial Baseada em Espécies (\textit{Species-based DE} - SDE) \cite{li2005efficient}: Seguindo os passos para determinar sementes de espécies (centros de populações menores dentro da população original), o SDE é capaz de identificar múltiplas espécies entre toda a população a cada iteração. Cada espécie identificada é otimizada por uma instância do DE.

\item Evolução Diferencial Baseada em Compartilhamento (\textit{Sharing-based DE} - ShDE) \cite{thomsen2004multimodal}: O SDE modifica o DE convencional da maneira seguinte. Em primeiro lugar, em vez dos pais serem substituídos por todas as crias, elas são adicionados à população. Em segundo lugar, a adequação de todos os indivíduos é redimensionada usando a função de \textit{Sharing}. Em terceiro lugar, a população é ordenada no que diz respeito ao novo \textit{fitness}. Finalmente, a pior metade da população (igual ao tamanho da população inicial) é removido.
\end{itemize}

Após a análise dos algoritmos é demonstrado a proposta do autor, em que pode-se ver os pontos que foram adaptados dos trabalhos citados anteriormente e quais aspectos desses algoritmos foram alterados. No novo modelo, chamado de Evolução Diferencial Local à Base de Aglomeração e com Memória Baseada em Espécies (\textit{Crowding-based local Differential Evolution with Speciation-based Memory} - ClDES) existem 3 características que se diferenciam do DE convencional, sendo elas:

\begin{enumerate}
\item Mutação por Vizinhança: durante a mutação feita pelo ClDES são escolhidos somente os indivíduos que estão próximos a vizinhança, de forma que, em relação a população total, somente uma pequena parte pode ser escolhida.

\item Função de Teste: É estipulada uma função de teste para detectar mudanças no ambiente que não faz parte do processo evolutivo, porém a cada geração essa função é testada e se for identificado alguma alteração no resultado do \textit{fitness} fica comprovado a mudança no ambiente. Ao ser identificada essa mudança o algoritmo toma as ações necessárias para manter um determinado nível de \textit{fitness}.

\item Arquivo de Memória Baseada em Especiação: ao detectar uma mudança no ambiente, o sistema procura os núcleos de espécies, chamados de sementes de espécies, e a partir dele são selecionados indivíduos que serão mantidos na nova população. A nova população feita utilizando metade dos indivíduos selecionados pelas sementes de espécies e a outra metade é gerada aleatoriamente.
\end{enumerate}

O ClDES foi aplicado no MP \textit{benchmark} e foi comparado com outras versões do PSO, que são: \textit{Dynamic Species-Based Particle Swarm Optimizer} (DSPSO) e \textit{Clustering Particle Swarm Optimizer} (CPSO), em que mostra-se superior a ambas as versões em quase todos os casos. A principal dificuldade encontrada pelo algoritmo acontece quando o número de dimensões aumenta, tornando cada vez mais difícil do algoritmo sempre encontrar todos os picos do problema.

\section{Algoritmo de Vaga-Lumes baseado em multi-enxames - MSFA}
\label{sec:fa_behaviour}

O trabalho de \cite{farahani2011multiswarm} propões uma junção da ideia do FA com a ideia de multi-enxames (\textit{MultiSwarm} - MS). A ideia principal da técnica MS é dividir a população em um número de sub enxames, com o objetivo de posicionar cada um desses sub enxames sobre diferentes áreas, visando encontrar diferentes picos no espaço de busca. No entanto, simplesmente separar o enxame em um número de enxames independentes não são susceptíveis de serem eficazes, uma vez que não tenha interação entre os sub enxames. Existem algumas abordagens chamadas de Exclusão e Anti-convergência para resolver este problema.

\begin{enumerate}
\item Exclusão: A exclusão é uma interação local entre enxames próximos de colidir. Se um enxame é dividido em um certo número de sub enxames, pode acontecer que partículas de diferentes enxames girem em torno de um único pico. Isso é indesejável uma vez que a motivação por trás de uma abordagem MS é postular diferentes enxames em diferentes picos. A fim de evitar isso é feito uma competição entre os enxames, quem possuir o melhor \textit{fitness} continua na otimização, o outro sub enxame é extinto e é gerado novamente.

\item Anti-convergência: Anti- convergência é uma partilha de informações de cada interação entre todos os sub enxames como uma interação global no algoritmo MS, com o objetivo de permitir que novos picos sejam detectados.
\end{enumerate}

O algoritmo proposto usa a MS para localizar todos o picos do MP \textit{benchmark}, em que cada sub enxame é um FA. Para a manutenção da diversidade no sistema MS utiliza-se partículas carregadas e partículas quânticas. Devido à lenta velocidade de convergência do FA e as armadilhas em vários locais ótimos do espaço de busca, neste trabalho, um novo comportamento é introduzido que melhora o desempenho do FA. No algoritmo proposto todas as partículas quânticas de cada enxame irão se mover em direção do melhor vaga-lume global ($G_best$) se não houver nenhum vaga-lume melhor entre seus seus vizinhos. Este comportamento melhora a velocidade de convergência. Também para cobrir eventuais desvios no movimento de vaga-lume, é usado um ângulo que torna o movimento dos vaga-lumes previsíveis e dá uma direção para cada vaga-lume.

Os resultados obtidos da aplicação do Algoritmo de Vaga-Lumes Baseado em Multi-Enxames (\textit{Multiswarm Based Firefly Algorithm} - MSFA) no MP \textit{benchmark} mostram que o MSFA mostra-se superior ao FA e a outras versões do PSO. A principal característica constatada é quando o número de picos é diferente do número de sub enxames, de forma que se o número de sub enxames for menor o desempenho do algoritmo cai drasticamente.

\section{Algoritmo PSO em ambientes dinâmicos}
\label{sec:pso_behaviour}

O PSO é um dos algoritmos de SI mais estudados na área de otimização de problemas dinâmicos, de forma que são geradas várias versões dele com vários operadores interessantes \cite{carlisle2002applying}.

\subsection{\textit{Dynamic Species-Based Particle Swarm Optimizer} - DSPSO}
\label{sec:dspso}

O DSPSO \cite{parrott2006locating} é baseado na sua versão estacionária, o SPSO que é baseado no sistema de espécies. Um sistema baseado em espécies possui vários nichos de espécies e em cada um deles existe um centro geográfico chamado de semente da espécie (textit{specie seed}). Esse nichos de espécies são utilizados para que o sistema possa encontrar vários pontos ótimos durante o processo de otimização, fazendo com o que sistema não fique estagnado em um pico sub ótimo.

Para encontrar os pontos ótimos após uma mudança no ambiente, para cada partícula da população é reavaliado o \textit{fitness} em função do melhor \textit{fitness} da população. Então cada partícula usa sua informação de aptidão para realizar o movimento, tendo sempre uma noção prévia do ambiente ao seu redor. Para incentivar o enxame a realizar uma busca no ambiente e evitar uma convergência prematura do sistema, foi introduzido o parâmetro de limite de partículas por espécies ($P_{max}$), de forma que se uma partícula fosse entrar para um espécie mas o limite de integrantes daquela espécie exceder, a partícula com o menor valor de \textit{fitness} é reiniciada. Uma desvantagem desse sistema é que gera uma lentidão no processo de convergência se o $P_{max}$ for muito baixo.

Nos resultados pôde-se observar que nem sempre o DSPSO obteve resultados mais precisos, porém sempre foi capaz de encontrar mais ou o mesmo número de picos que os outros algoritmos.

\subsection{\textit{Clustering Particle Swarm Optimizer} - CPSO}
\label{sec:cpso}

Outra versão do PSO é o \textit{Clustering Particle Swarm Optimizer} CPSO \cite{yang2010clustering}, em que também utiliza o modelo de multi-enxames explicado na Seção \ref{sec:fa_behaviour}.

CPSO começa a partir de um enxame inicial, chamado de enxame berço. Em seguida, sub enxames são criados por um método de clusterização hierárquico. Quando sub enxames são criados, uma busca local é aplicada sobre eles, a fim de explorar potenciais picos próximos deles. Após isso é feita uma checagem de sub enxames sobrepostos, convergência prematura, e superlotação antes dos próximos ciclos de iteração. Se uma mudança ambiental é detectada, um novo enxame berço será aleatoriamente re-gerado com a reserva das posições dos sobreviveram dos sub enxames anteriores.

A divisão do enxame berço para os sub enxames é feita por um processo de clusterização hierárquico de ligação única. O processo de busca local utiliza a partícula de melhor \textit{fitness} global para direcionar o enxame berço, a fim de obter uma convergência mais rápida. Esse modelo de busca local é o utilizado na versão normal do PSO. 

O resultados do algoritmo aplicados no \textit{benchmark} MP mostram que ele teve um performance superior as outras versões do PSO analisadas, de forma que quase sempre foi possível encontrar todos os picos e o resultado ótimo.

\subsection{\textit{Volitive Particle Swarm Optimizer} - VPSO}
\label{sec:vpso}

Uma aplicação do PSO gerou uma versão hybrida dele com o FSS, utilizando os operadores de alimentação e de movimentação coletiva volátil. No \textit{Volitive Particle Swarm Optimizer} (VPSO) \cite{cavalcanti2011hybrid}, cada partícula torna-se uma partícula ponderada, em que o peso é usado para calcular o movimento colessstivo volátil, resultando na expansão ou contração do enxame. Nessa proposta o $step_{vol}$ não diminui linearmente, ele decresce a partir da Equação \ref{eq:vpso}. A percentagem de decaimento do parâmetro volátil ($decay_{vol}$) deve estar no intervalo [0,100].

\begin{equation}
\label{eq:vpso}
step_{vol}(t+1) = step_{vol}(t) \frac{100 - decay_{vol}}{100}
\end{equation}

\noindent O $step_{vol}$ é reinicializado para o $step_{volMax}$ quando uma mudança no ambiente é detectado. É utilizada uma partícula sentinela para detectar estas mudanças. A aptidão da partícula sentinela é avaliada no final de cada iteração e no início da iteração seguinte. Este Algoritmo foi aplicado no DF1 \textit{benchmark} e obeteve resultados melhores que suas versões anteriores, mostrando assim a grande influência que o operador de movimento volátil tem sobre os ambientes dinâmicos.

\section{Otimização por Colônia de Bactérias em Problemas Dinâmicos}
\label{sec:bfo_behaviour}

O processo de reprodução da BFA que tem o objetivo de acelerar a convergência é adequado em problemas estáticos, porém isso gera uma falta de adaptação em ambientes dinâmicos. Assim, a fim de se comprometer entre a convergência rápida e a alta diversidade, é proposto um algoritmo dinâmico de Otimização por Colônia de Bactérias (\textit{Dynamic Bacterial Foraging Algorithm} - DBFA) \cite{passino2002biomimicry} em que não é utilizada a eliminação-dispersão. Um processo de seleção é introduzido através de um esquema mais flexível para permitir uma melhor capacidade de adaptação em um ambiente em mudança. A ideia básica do DBFA é manter uma diversidade adequada para pesquisa global, enquanto a capacidade de busca local não é degradada, e também considerar as alterações no ambiente. O processo de seleção é descrito na Equação \ref{eq:dbfo}

\begin{equation}
\label{eq:dbfo}
\begin{split}
& J_i = \sum_{j=1}^{n} J_i(j,r) \\
& rank_i = sort(J_i) \\
& W_i = m \frac{(rank_i)^k}{\sum_{i=1}^{P} (rank_i)^k} + (1 - m) \frac{J_i}{\sum_{i=1}^{P} J_i}
\end{split}
\end{equation}

\noindent em que $n$ é o número de etapas quimiotáticos (cada passo pode conter uma corrida ou um tombo) durante o tempo de vida de uma bactéria, $j$ é o seu índice e $P$ é o tamanho da população, o símbolo $m$ representa o peso de diversidade, e $k$ é o expoente de classificação $rank_i$. Assim, essas áreas mais ricas em nutrientes experientes (melhor \textit{fitness}) são mais propensas a serem selecionadas como um pai para a próxima geração. No entanto, esse domínio não ajudaria na manutenção da diversidade.

Em seguida tem-se a combinação da solução e da classificação, que são escolhidas para evitar uma rápida convergência e devem ser evitadas para manter uma capacidade de adaptação do DBFA. Assim, toda a população é classificada de acordo com $J_i$ usando um tipo de operador, em seguida, $rank_i$ é alocado como classificação da bactéria $i$. É utilizado o parâmetro $m$ que afeta a diversidade para o processo de seleção através da combinação da classificação da bactéria ($rank_i$) $k$ com o cálculo de \textit{fitness} $J_i$. A probabilidade de sobrevivência da bactéria é determinada pela somatória da variável $W_i$. Ao final é aplicada a seleção por roleta utilizada nos AGs.

O BFA e o DBFA foram aplicados no \textit{benchmark} MP e foram avaliados seus desempenho e sua diversidade durante a otimização. Foi constatado que a diversidade de DBFA muda depois de cada processo quimiotático em vez da dispersão adotada pelo BFA, que ocorre depois de várias gerações. O DBFA utiliza não só a busca local, mas também aplica o esquema de seleção flexível para manter uma diversidade apropriada durante todo o processo evolutivo. O DBFA supera o BFA em quase todos os ambientes dinâmicos. Além disso, a detecção de mudanças ambientais não é necessária no DBFA. O DBFA tem a mesma complexidade computacional que a do BFA , mas oferece o melhor desempenho.

\section{Considerações}
\label{sec:problems_considerations}

Com a análise destes problemas foi possível extrair algumas informações relevantes, que estão esquematizadas na Tabela \ref{tab:problems_table}, entre elas tem-se os problemas abordados pelos algoritmos, seuas rotinas de manutenção da diversidade populacional e por final é apresentado alguma outras informações importântes de cada problema em particular.

O CIDES utilisa como conceito as suas versões anteriores, para que assim podesse diminuir a quantidade de erro da sua aplicação em problemas dinâmicos. A sua mutação por vizinhança ajudou a manter a busca global mais controlada, de forma que pudesse buscar por valores mais próximos de forma aleatória. Sua função de teste é uma das formas mais básicas estudades de perceber mudanças no ambiente, o que é uma ferramente muito importante na aplicação dos algoritmos em ambiente que sofrem essas mudanças constantemente.

O MSFA mostra a importância da aplicação da abordagem de Multi Enxames, que tem como base a co-evolução. suas rotinas de exclusão e anti-convergência formam a comunicação entre as pequenas populações que existem na otimização, abstraindo a otimização um nível a cima de apenas uma população.

O VPSO demostra uma aplicação do operador de movimentação coletiva volátil do FSS, que é o foco da pesquisa deste trabalho, e mostra que tem grandes possibilidade para ajudar a manter a população no caminha certo durante as mudanças do ambiente.

No DBFA pode-se notar mais uma vesão conjunta de dois algoritmos, em que usa-se o Algoritmo BFA com a rotina de seleção do AG. Isso mostra que mesmo algortimos de classes diferentes (como os EA com o AG e os SI com o BFA) podem ser aplicados em conjunto e terem resultados melhores que separados.

\begin{table}[]
	\label{tab:problems_table}
	\centering
	\caption{Tabela das Relações dos Algoritmos Analisados}
	\label{my-label}
	\begin{tabular}{|c|c|c|}
		\hline
		Algoritmos & Problemas Abordados & Rotinas de manutenção da diversisdade \\ \hline
		CDE        & MP \textit{benchmark}   & \textit{Crowding} \\ \hline
		SES        & OP \textit{benchmark}   & Diferenciação por espécies \\ \hline
		ShDE       & MP \textit{benchmark}   & \textit{Sharing} \\ \hline
		CIDES      & MP \textit{benchmark}   & Mutação por Vizinhança \\ \hline
		MSFA       & MP \textit{benchmark}   & Parículas carregadas e Quânticas \\ \hline
		DSPSO      & MP \textit{benchmark}   & Limitação de partículas por Espécies \\ \hline
		CPSO       & MP \textit{benchmark}   & Clusterização Hierárquico de Ligação Única \\ \hline
		VPSO       & DF1 \textit{benchmark}  & Movimentação volátil \\ \hline
		DBFA       & MP \textit{benchmark}   & Combinação da Classificação das Bactérias \\ \hline
	\end{tabular}
\end{table}

Em geral a manutenção da diversidade da população sempre tem sido um dos focos no desenvolvimento de novas abordagens de algoritmos bioinspirados, com isso pode-se afirmar que nessa classe de problemas esse é um ponto crucial que deve ser levado em concideração e deve ser analisado várias técnicas diferentes para estipular a relevância de cada uma dela.