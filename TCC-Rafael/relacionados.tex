\chapter{Trabalhos Relacionados}
\label{ch:relacionados}
A Computação Natural possuí vários operadores diferentes em algoritmos diferentes para serem aplicados em diversos tipos de problemas. Nesta Seção são apresentados os trabalhos do estado da arte encontrados na literatura que utilizam esses algoritmos bioinspirados em problemas dinâmicos de domínio contínuo.

Os trabalhos apresentados nesta seção são utilizado para analisar a aplicação destes algoritmos em diferentes ambiente e com diferentes condições, conseguindo assim extrair informações de influência de cada operador evolutivo utilizado na otimização dos problemas e/ou re-otimização.

\section{Comportamento do AG em ambientes dinâmicos}
\label{sec:ag_behaviour}

No trabalho de \cite{rand2005measurements} é feito uma análise do comportamento do AG em ambientes dinâmicos e mostra que na maioria dos trabalhos que analisam o AG, somente a performance do algoritmo é analisada, ou seja, o quão perto do melhor resultado chegou. Então são analisados quatro fatores principais para determinar a eficiência do AG em ambientes dinâmicos neste trabalho, sendo eles:

\begin{enumerate}
\item Performance: Para entender a performance do algoritmo existe duas vertentes, sendo uma a avaliação do melhor indivíduo da população para cada iteração do AG, e a outra é avaliar a média da população em para cada uma das iterações. Para a aplicar AG no SL-HDF é usado a melhor solução antes da primeira alteração como média inicial.

\item Satisfabilidade: É a medida da habilidade do sistema de manter um certo nível de \textit{fitness} no decorrer da otimização e não deixar esse nível cair abaixo de um determinado limite. Esta medida não necessariamente representa o quão rápido (menos interações necessárias) o sistema chega em uma nova ótima solução, e sim se ele consegue manter um nível de \textit{fitness} da população.

\item Robustez: É a medida de como o sistema reage a uma alteração, de forma que ao sofrer uma alteração o \textit{fitness} não pode ter uma queda muito brusca. A medida de robustez usada neste trabalho foi a média do \textit{fitness} no estado atual do ambiente sobre a média do \textit{fitness} no estado anterior do sistema, para uma alteração perceptível.

\item Diversidade: É a medida que representa a variação do genoma da população, de forma que uma população que possuí uma alta diversidade tem maiores chances de encontrar novas solução e assim se adaptar melhor a uma mudança do ambiente. Existem várias técnicas estudadas para manter a diversidade da população durante o processo evolutivo, e para medir a diferença genotípica é usado a distância de \textit{Hamming}.
\end{enumerate}

Na aplicação do AG no SL-HDF pode-se notar que a performance do algoritmo no ambiente dinâmico é superior sua aplicação em ambientes estáticos quando há um grande número de iterações. Inicialmente o ambiente dinâmico perde para o estático mas a partir da metade do processo evolutivo o dinâmico gera um \textit{fitness} maior no melhor indivíduo e na média da população.

A análise da satisfabilidade mostra que em relação a aplicação em um ambiente estático o ambiente dinâmico tem quase o mesmo nível médio de \textit{fitness} porém o ambiente dinâmico está sendo recompensado por blocos de construção intermediária diferentes e, portanto, tem uma pressão seletiva superior para encontrá-los.

Na analise de robustez pode-se notar que a cada 100 iteração (quando ocorre uma mudança no ambiente) exite uma queda na média do \textit{fitness}, porém o sistema se recupera rapidamente, e a cada nova mudança e queda do \textit{fitness} diminui.

A diversidade do sistema teve um comportamento inesperado, pois inicialmente achava-se que o sistema iria perder a diversidade até uma mudança ocorrer e depois a diversidade iria aumentar, porém aconteceu exatamente o contrário, tendo que após uma mudança a diversidade diminui e vai aumentando até identificar uma nova mudança.

\section{Evolução Diferencial Local à Base de Aglomeração e com Memória Baseada em Espécies}
\label{sec:crowding_base_de}

No trabalho de \cite{kundu2013crowding}, é feito uma análise comparativa das versões do DE exitesntes, especificando suas necessidades ao serem aplicados em ambiente dinâmicos e ao final é apresentado a proposta de aplicação. As versões que são analisadas são as seguintes:

\begin{itemize}
\item Evolução Diferencial Baseada em Aglomeração (\textit{Crowding-based DE} - CDE) \cite{thomsen2004multimodal}: Basicamente, o CDE estende DE com o esquema \textit{Crowding}. Assim, a única modificação ao DE convencional é sobre o indivíduo (pai) ser substituído. Normalmente o pai produzir a descendência é substituído, ao passo que no CDE a descendência substitui o indivíduo mais similar entre um subconjunto da população.

\item Evolução Diferencial Baseada em Espécies (\textit{Species-based DE} - SDE) \cite{li2005efficient}: Seguindo os passos para determinar sementes de espécies (centros de populaçoes menores dentro da população origianl), o SDE é capaz de identificar múltiplas espécies entre toda a população a cada iteração. Cada espécie identificada é otimizada por uma instância do DE.

\item Evolução Diferencial Baseada em Compartilhamento (\textit{Sharing-based DE} - ShDE) \cite{thomsen2004multimodal}: O SDE modifica o DE convencional da maneira seguinte. Em primeiro lugar, em vez dos pais serem substituidaos por todas as crias, elas são adicionados à população. Em segundo lugar, a adequação de todos os indivíduos é redimensionada usando a função de \textit{Sharing}. Em terceiro lugar, a população é ordenada no que diz respeito ao novo \textit{fitness}. Finalmente, a pior metade da população (igual ao tamanho da população inicial) é removido.
\end{itemize}

Após a análise dos algoritmos é demonstrado a proposta do autor, em que pode-se ver os pontos que foram adaptados dos trabalhos citados anteriormente e quais aspéctos desses algoritmos foram alterados. No novo modelo, chamado de Evolução Diferencial Local à Base de Aglomeração e com Memória Baseada em Espécies (\textit{Crowding-based local Differential Evolution with Speciation-based Memory} - ClDES), possui 3 características que se diferenciam do DE convencional, sendo elas:

\begin{enumerate}
\item Mutação por Vizinhança: durante a mutação feita pelo ClDES são escolhidos somente os indivíduos que estão próximos de quem será mutado, de forma que, em relação a população total, somente uma pequena parte pode ser escolhida.

\item Função de Teste: É estipulada uma função de teste para detectar mudanças no ambiente que não faz parte do processo evolutivo, porém a cada geração essa função é testada e se for identificado alguma alteraçao no resultado do \textit{fitness} fica comprovado a mudança no ambiente. Ao ser identificado essa mudança o algoritmo toma as açoes necessárias para manter um determinado nível de \textit{fitness}.

\item Arquivo de Memória Baseada em Especiação: ao detectar uma mudança no ambiente, o sistema procura os núcleos de espécias, chamados de sementes de espécies, e a partir dele são selecionados indivíduos que serão mantidos na nova população. A nova população feita utilizando metade dos indivíduos selecionados pelas sementes de espécies e a outra metade é gerada aleatóriamente.
\end{enumerate}

O ClDES foi aplicado no MP \textit{benchmark} e foi comparado com outras versões do PSO, que são: \textit{Dynamic Species-Based Particle Swarm Optimizer} (DSPSO) e \textit{Clustering Particle Swarm Optimizer} (CPSO), em que mostra-se superior a ambas as versões em quase todos os casos. A principal dificuldade encontrada pelo algoritmo acontece quando o número de dimensões aumenta, tornando cada vez mais difícil do algoritmo sempre encontrar todos os picos do problema.

\section{Comportamento do PSO em ambientes dinâmicos}
\label{sec:pso_behaviour}
O PSO\cite{carlisle2002applying}
\textit{Volitive Particle Swarm Optimizer} VPSO \cite{cavalcanti2011hybrid}
\textit{Dynamic Species-Based Particle Swarm Optimizer} DSPSO \cite{parrott2006locating}
\textit{Clustering Particle Swarm Optimizer} CPSO \cite{yang2010clustering}

\section{Algoritmo Dinâmico de Otimização por Colônia de Bactérias}
\label{sec:bfo_behaviour}

O processo de reprodução de BFA com o objetivo de acelerar a convergência é adequado em problemas estáticos, porém isso gera uma falta de adaptação em ambientes dinâmicos. Assim, a fim de se comprometer entre a convergência rápida e a alta diversidade, é propomos um algoritmo dinâmico de Otimização por Colônia de Bactérias (\textit{Dynamic Bacterial Foraging Algorithm} - DBFA) \cite{passino2002biomimicry} em que não é utilizado a eliminação-dispersão, um processo de seleção é introduzido através de um esquema mais flexível para permitir uma melhor capacidade de adaptação em um ambiente em mudança. A ideia básica da DBFA é manter uma diversidade adequada para pesquisa global, enquanto a capacidade de busca local não é degradada, e também considerar as alterações no ambiente. O processo de seleção é descrito na Equação \ref{eq:dbfo}

\begin{equation}
\label{eq:dbfo}
\begin{split}
& J_i = \sum_{j=1}^{n} J_i(j,r) \\
& rank_i = sort(J_i) \\
& W_i = m \frac{(rank_i)^k}{\sum_{i=1}^{P} (rank_i)^k} + (1 - m) \frac{J_i}{\sum_{i=1}^{P} J_i}
\end{split}
\end{equation}

\noindent em que $n$ é o número de etapas quimiotácticos (cada passo pode conter uma corrida ou um tombo) durante o tempo de vida de uma bactéria, $j$ é o seu índice e $P$ é o tamanho da população, o símbolo $m$ representa o peso de diversidade, e $k$ é o expoente de classificação $rank_i$. Assim, essas áreas mais ricas em nutrientes experientes são mais propensas a ser selecionadas como um pai para a próxima geração. No entanto, esse domínio não ajudaria a diversidade manutenção.

Em seguida tem-se a combinação da solução e da classificação, que são escolhida para evitar uma rápida convergência e devem ser evitada para manter uma capacidade de adaptação do DBFA. Assim, toda a população é classificada de acordo com $J_i$ usando um tipo de operador, em seguida, $rank_i$ é alocado como classificação da bactéria $i$. É utilizado o parâmetro $m$ que afeta a diversidade para o processo de selecção através da combinação da classificação da bactéria ($rank_i$) $k$ com o cálculo de \textit{fitness} $J_i$. A probabilidade de sobrevivência da bactéria é determinada pela somatória da variável $W_i$. Ao final a aplicado a seleçao por roleta utilizada nos AGs.

O BFA e o DBFA foram aplicados no MP \textit{benchmark} e foi avaliado seus desempenho e sua diversidade durante a otimização. Foi constatado que a diversidade de DBFA muda depois de cada processo quimiotático em vez da dispersão adotada pelo BFA, que ocorre depois de várias gerações. O DBFA utiliza não só a busca local, mas também aplica o esquema de selecção flexível para manter uma diversidade apropriada durante todo o processo evolutivo. O DBFA supera o BFA em quase todos os ambientes dinâmicos. Além disso, a detecção de mudanças ambientais não é necessário no DBFA. O DBFA tem a mesma complexidade computacional com a do BFA , mas oferece o melhor desempenho.

\section{Algoritmo de Vaga-Lumes baseado em milti-enxames}
\label{sec:fa_behaviour}

O trabalho de \cite{farahani2011multiswarm} propões uma junção da ideia do FA com a ideia de multi-enxames (\textit{MultiSwarm} - MS). A idéia principal da técnica MS é dividir a população em um número de sub enxames, com o objetivo de posicionar cada um desses sub enxames sobre diferentes áreas, visando encontrar diferentes picos no espaço de busca. No entanto, simplesmente separar o enxame em um número de enxames independentes não são susceptíveis de serem eficazes, uma vez que não tenha interação entre os sub enxams. existem algumas abordagens chamadas de Exclusão e Anti-convergência para resolver este problema.

\begin{enumerate}
\item Exclusão: A exclusão é uma interação local entre enxames próximos de colidir. Se um enxame é dividido em um certo número de sub enxames, pode acontecer que partículas de diferentes enxames girem em torno de um único pico. Isso é indesejável uma vez que a motivação por trás de uma abordagem MS é postular diferentes enxames em diferentes picos. A fim de evitar isso é feito uma competição entre os enxames, quem possuir o melhor \textit{fitness} continua na otimização, o outro sub enxame é extinto e é gerado novamente.

\item Anti-convergência: Anti- convergência é uma partilha de informações interações entre todos os sub enxames como uma interação global no algoritmo MS, com o objetivo de permitir que novos picos sejam detectados.
\end{enumerate}

O algoritmo proposto usa a MS para localizar todos o picos do MP \textit{benchmark}, em que cada sub enxame é um FA. Para a manutenção da divesidade no sietema MS utiliza-se partículas carregadas e partículas quâmticas. Devido à lenta velocidade de convergência do FA e as armadilhas em vários locais ótimos do espaço de busca, neste trabalho, um novo comportamento é introduzido que melhora o desempenho do FA. No algoritmo proposto todas as partículas quânticas de cada enxame irão se mover em direção do melhor vaga-lume global ($G_best$) se não houver nenhum vaga-lume melhor entre seus seus vizinhos. Este comportamento melhora a velocidade de convergência. Também para cobrir eventuais desvios no movimento de vaga-lume, é usado um ângulo que torna o movimento dos vaga-lumes previsíveis e dá uma direção para cada vaga-lume.

Os resultados obtidos da aplicação do Algoritmo de Vaga-Lumes Baseado em Multi-Enxames (\textit{Multiswarm Based Firefly Algorithm} - MSFA) no MP \textit{benchmark} mostram que o MSFA mostra-se superior ao FA e a outras versões do PSO. A principal característica constatada é quando o número de picos é diferente do número de sub enxames, de forma que se o número de sub enxames for menor o desempenho do algoritmo cai drasticamente.