\chapter{Trabalhos Relacionados}
\label{ch:relacionados}
A Computação Natural possuí vários operadores diferentes em algoritmos diferentes para serem aplicados em diversos tipos de problemas. Nesta Seção são apresentados os trabalhos do estado da arte encontrados na literatura que utilizam esses algoritmos bioinspirados em problemas dinâmicos de domínio contínuo.

Os trabalhos apresentados nesta seção são utilizado para analisar a aplicação destes algoritmos em diferentes ambiente e com diferentes condições, conseguindo assim extrair informações de influência de cada operador evolutivo utilizado na otimização dos problemas e/ou re-otimização.

\section{Comportamento do AG em ambientes dinâmicos}
\label{sec:ag_behaviour}

No trabalho de \cite{rand2005measurements} é feito uma análise do comportamento do AG em ambientes dinâmicos e mostra que na maioria dos trabalhos que analisam o AG, somente a performance do algoritmo é analisada, ou seja, o quão perto do melhor resultado chegou. Então são analisados quatro fatores principais para determinar a eficiência do AG em ambientes dinâmicos neste trabalho, sendo eles:

\begin{enumerate}
\item Performance: Para entender a performance do algoritmo existe duas vertentes, sendo uma a avaliação do melhor indivíduo da população para cada iteração do AG, e a outra é avaliar a média da população em para cada uma das iterações. Para a aplicar AG no SL-HDF é usado a melhor solução antes da primeira alteração como média inicial.

\item Satisfabilidade: É a medida da habilidade do sistema de manter um certo nível de \textit{fitness} no decorrer da otimização e não deixar esse nível cair abaixo de um determinado limite. Esta medida não necessariamente representa o quão rápido (menos interações necessárias) o sistema chega em uma nova ótima solução, e sim se ele consegue manter um nível de \textit{fitness} da população.

\item Robustez: É a medida de como o sistema reage a uma alteração, de forma que ao sofrer uma alteração o \textit{fitness} não pode ter uma queda muito brusca. A medida de robustez usada neste trabalho foi a média do \textit{fitness} no estado atual do ambiente sobre a média do \textit{fitness} no estado anterior do sistema, para uma alteração perceptível.

\item Diversidade: É a medida que representa a variação do genoma da população, de forma que uma população que possuí uma alta diversidade tem maiores chances de encontrar novas solução e assim se adaptar melhor a uma mudança do ambiente. Existem várias técnicas estudadas para manter a diversidade da população durante o processo evolutivo, e para medir a diferença genotípica é usado a distância de \textit{Hamming}.
\end{enumerate}

Na aplicação do AG no SL-HDF pode-se notar que a performance do algoritmo no ambiente dinâmico é superior sua aplicação em ambientes estáticos quando há um grande número de iterações. Inicialmente o ambiente dinâmico perde para o estático mas a partir da metade do processo evolutivo o dinâmico gera um \textit{fitness} maior no melhor indivíduo e na média da população.

A análise da satisfabilidade mostra que em relação a aplicação em um ambiente estático o ambiente dinâmico tem quase o mesmo nível médio de \textit{fitness} porém o ambiente dinâmico está sendo recompensado por blocos de construção intermediária diferentes e, portanto, tem uma pressão seletiva superior para encontrá-los.

Na analise de robustez pode-se notar que a cada 100 iteração (quando ocorre uma mudança no ambiente) exite uma queda na média do \textit{fitness}, porém o sistema se recupera rapidamente, e a cada nova mudança e queda do \textit{fitness} diminui.

A diversidade do sistema teve um comportamento inesperado, pois inicialmente achava-se que o sistema iria perder a diversidade até uma mudança ocorrer e depois a diversidade iria aumentar, porém aconteceu exatamente o contrário, tendo que após uma mudança a diversidade diminui e vai aumentando até identificar uma nova mudança.
\section{Comportamento do PSO em ambientes dinâmicos}
\label{sec:pso_behaviour}
O PSO\cite{carlisle2002applying}

\cite{modified_pso}
