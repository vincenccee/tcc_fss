\titulo{Análise do Algoritmo de Cardume de Peixes em Problemas Dinâmicos com Domínio Contínuo}
\autor{Rafael Henrique Vincence}
\nome{Rafael Henrique}
\ultimonome{Vincence}

\bacharelado \curso{Ciência da Computação}
\ano{2016}
\data {\today}
\cidade{Joinville}

\instituicao{Universidade do Estado de Santa Catarina}
\sigla{UDESC}
\unidadeacademica{Centro de Ciências Tecnológicas}

\orientador{Profº Rafael Stubs Parpinelli}
\examinadorum{Profº Guilherme Koslovski}
\examinadordois{Profº Chidambaram Chidambaram}

\ttorientador{$ $}
\ttexaminadorum{$ $}
\ttexaminadordois{$ $}

\newpage
\pagestyle{empty}

\maketitle

%\begin{dedicatoria}
%\noindent
%\end{dedicatoria}
%\noindent
%\newpage
%\begin{epigrafe}
%\noindent
%"xxxxxxx"
%--alguem
%\end{epigrafe}

%\agradecimento{Agradecimentos}
%caneta

\resumo{Resumo}

A maioria dos desafios encontrados na vida real são problemas complexos com uma grande gama de variáveis e por vezes dinâmico. Sendo assim, cada vez mais tem-se a necessidade da criação de meta-heurísticas para solucionar problemas com estas características. A Natureza, por sua vez, resolve instintivamente esses tipos de problemas e por esse motivo algoritmos bioinspirados têm sido amplamente utilizados na resolução de problemas dinâmicos com domínio contínuo. Os aspectos e as interações de uma colônia de animais são pontos relevantes na otimização desses problemas, como o comportamento individual e comunitário, que contribuem em problemas de larga escala e não estacionários. Neste trabalho o foco é a análise e aplicação do algoritmo de cardume de peixes (\textit{Fish School Search}, FSS) em \textit{benchmarks} dinâmicos de domínio contínuo.

\noindent \textbf{Palavras-chave:} \textit{Inteligência de enxames}, \textit{Algoritmos bioinspirados}, \textit{Problemas dinâmicos com domínio contínuo}, \textit{Fish School Search}.

\tableofcontents
\listoffigures
\listoftables
\newpage
\chapter*{Lista de Abreviaturas\hfill} \addcontentsline{toc}{chapter}{Lista de Abreviaturas}
\listofsymbols

\newpage
\pagestyle{myheadings}

